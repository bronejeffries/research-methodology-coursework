\documentclass[11pt]{article}
%Gummi|062|=)
\title{\textbf{MAKERERE UNIVERSITY \\FACULTY OF COMPUTING AND INFORMATICS\\DEPARTMENT OF COMPUTER SCIENCE.}}
\author{SSEMATE BRIAN\\21600206\\16/U/1134.}
\date{}
\begin{document}

\maketitle

\title{\textbf{HOW CAN TRAFFIC CONGESTION BE REDUCED ON MOST ROADS IN UGANDA?}}

This study about Traffic jam has been carried out in Uganda, Kampala in particular.

 A lot happens to the motor drivers in traffic and more is still happening due to the fact that there is nothing they can do about it other than moving so early whenever they are moving around Kampala in order to reach in time. At this moment some roads are so commonly known for their high levels of traffic, which worries the drivers at times and in the end they try their best to avoid using these roads as another measure not to be stack in the congestions. 

The following data entries have been used during the questioning part of the study; Name, gender, physical address, phone no., problem faced due to traffic congestion, road on which it happened, suggestion on how to reduce traffic congestion, comment on the study exercise.\\


\title{\textbf{CAUSES OF TRAFFIC CONGESTION.}}\\

The main cause of traffic congestion is the big number of vehicles using the small sized roads in Kampala yet increasing their size is now not an option we can take (it’s somehow difficult).\\

Failure to apply and respect rules that govern traffic; due to the unstrict law enforcement body, drivers most times feel free to unrespect the rules that are put up to be followed.\\


\title{\textbf{OBJECTIVES OF THE RESEARCH.}}\\

Reduce traffic congestion on Kampala roads and hopefully countrywide: this can be done in very many ways both by the use of technology and usage of traffic rules and regulation like setting speed limits on roads, and the usage of traffic lights (already in place but not really enough according to the number of road users and the number of roads being used). By an advancement in technology we can find a way of alerting drivers so that they can dodge the congestion earlier. The less number of vehicles in a congestion the easier it can be controlled.\\

To find ways on how to increase roads’ efficiency: Roads most times are considered as a sign of development but also at the same time they happen to be a serious cause of chaos in urban places which happen to have many motor drivers (and numbers increase so fast) due to the traffic congestion .\\

Reduce the dangers caused by traffic congestion: Traffic congestion is also a parent to various dangers. The road users who happen to get caught by the traffic congestion face a lot of problems like theft, being delayed to work, for appointments which are important to them. These mentioned problems can be stopped if traffic congestion is stopped and then give peace to the road users. 

\title{\textbf{SAMPLE DATA COLLECTED.}}\\
\textbf{Name:} Mr.Edward \\\textbf{Gender:} Male \\\textbf{Address:} Wandegeya\\ \textbf{PhoneNo.:} 0756744730\\ \textbf{problem faced:} missed appointment\\\textbf{Road Name:} Entebbe road\\\textbf{GPS Co-ordinates:}‎ 0.347596,‎32.582520\\\textbf{suggestion:} Build emergency roads\\\textbf{comment on the exercise:} helpful\\



\end{document}
